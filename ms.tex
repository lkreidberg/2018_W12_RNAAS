\documentclass[RNAAS]{aastex62}

\newcommand{\vdag}{(v)^\dagger}
\newcommand\aastex{AAS\TeX}
\newcommand\latex{La\TeX}

\submitjournal{RNAAS}

\shorttitle{Non-Detection of Excited Helium for WASP-12b}
\shortauthors{Kreidberg \& Oklop\v{c}i\'{c}}

\begin{document}

\title{A Non-Detection of Excited Helium in the Atmosphere of the Evaporating Hot Jupiter WASP-12b}


\author{Laura Kreidberg}
\affiliation{Harvard-Smithsonian Center for Astrophysics, 60 Garden Street, Cambridge, MA 02138}
\affiliation{Harvard Society of Fellows, 78 Mount Auburn Street, Cambridge, MA 02138}

\author{Antonija Oklop\v{c}i\'{c}}
\affiliation{Harvard-Smithsonian Center for Astrophysics, 60 Garden Street, Cambridge, MA 02138}
Excited helium was recently detected escaping the atmosphere of the exoplanet WASP-107b, a low-density, warm Neptune \citep{spake18}. Absorption features from the decay of metastable helium at $10833\AA$ were predicted by \cite{seager00} and \cite{oklopcic18}. The helium line provides a new probe of atmospheric escape that is advantageous in several ways: (1) it is observable with near-infrared facilities (in contrast to other signposts of atmospheric escape that appear in the ultraviolet) and (2) it is minimally affected by interstellar absorption, thereby opening the door to studying atmospheric escape in a greater number of systems.

Inspired by WASP-107b detection, we searched archival HST observations of another evaporating exoplanet, WASP-12b, for signs of helium. WASP-12b is a good candidate for this search: it is one of the most inflated and hottest known hot Jupiters \citep[$R_p = 1.79\,R_\mathrm{J}$, $T_\mathrm{eq} = 2500$ K][]{hebb09} At this level of intense irradiation, theory predicts a high rate of escaping atoms and molecules from the planet's atmosphere, and indeed, transit observations in the ultraviolet have revealed a patchy cloud of escaping material \citep{haswell12, nichols15}.  

For this Note, we reanalyzed three transits observations of WASP-12b from the
Hubble Space Telescope/Wide Field Camera 3 \citep[originally published in
][]{kreidberg15b}. These observations used the G102 grism, which spans the
$10833\,\AA$ helium feature. In our analysis, we used the same methodology as \cite{kreidberg15b}, except with different spectral binning to include a narrow band ($70\AA$, the spectrograph's native resolution) centered on the He feature.  
We allowed the transit depth to vary from visit to visit, but found no evidence
for variability FIXME. The final transmission spectrum

Figure 1 shows the resulting transmission spectrum. 

The exact EUV spectrum of the host star is unknown. However, there is evidence
from optical activity indicators that

To estimate the expected absorption signal of WASP-12b at 10830 \AA, we use the theoretical model described in \cite{oklopcic18}, slightly modified for this purpose. In this 1D model, the upper atmosphere (thermosphere) of the planet is assumed to be composed of atomic hydrogen and helium in 9:1 number ratio. For the thermospheric density and velocity profiles we adopt the isothermal Parker wind model, assuming the gas temperature of $T=10^4$~K and the total atmospheric mass loss rate of $4\times 10^{11}$~g~s$^{-1}$. These values are based on the results of hydrodynamic simulations of atmospheric escape in WASP-12b by \cite{salz16}. Furthermore, we use the solar irradiance spectrum as the input spectrum. 

Under these assumptions, the calculated population of metastable helium within the Roche radius of the planet produces very little absorption at 10830 \AA, consistent with the observed non-detection. However, theoretical results are highly dependent on the assumed geometry and parameters of the model, as well as the input stellar spectrum. Although WASP-12 and the Sun are stars of similar spectral types, WASP-12 shows a much lower level of stellar activity (Haswell 2017), and hence its EUV flux, responsible for populating the metastable helium state, might differ considerably from that of the Sun. 

point us to the conclusion that either the gas is dispersed, the stellar spectrum is faint at EUV wavelengths, or both. These factors should be considered in future searches for escaping helium gas.


\bibliographystyle{aasjournal}
\bibliography{ms.bib}

\end{document}

% End of file `sample62.tex'.
